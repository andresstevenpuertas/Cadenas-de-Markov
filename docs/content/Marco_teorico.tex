\section{Introducción}

Este documento presenta la implementación de dos modelos de cadenas de Markov utilizando el algoritmo Gibbs Sampler:

\subsection{Modelo Hard-Core}

El modelo Hard-Core representa configuraciones de partículas en una rejilla donde ninguna pareja de partículas puede ser adyacente. Una configuración válida cumple que para cada par de vértices adyacentes, al menos uno está vacío.

El objetivo es generar muestras de configuraciones factibles distribuidas uniformemente, calculando estadísticas como:
\begin{itemize}
\item Número promedio de partículas
\item Densidad de ocupación
\item Convergencia de la distribución
\end{itemize}

\subsection{Modelo q-Coloraciones}

El modelo de q-coloraciones asigna colores a los vértices de una rejilla de manera que vértices adyacentes tengan colores diferentes. Este problema es fundamental en teoría de grafos.

El objetivo es generar coloraciones propias distribuidas uniformemente y analizar:
\begin{itemize}
\item Distribución de vértices por color
\item Verificación de coloraciones válidas
\item Convergencia a la distribución estacionaria
\end{itemize}

